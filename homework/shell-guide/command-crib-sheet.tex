\documentclass[12pt, oneside]{amsart}   	% use "amsart" instead of "article" for AMSLaTeX format
\usepackage{geometry}                		% See geometry.pdf to learn the layout options. There are lots.
\geometry{letterpaper}                   		% ... or a4paper or a5paper or ... 
%\geometry{landscape}                		% Activate for for rotated page geometry
%\usepackage[parfill]{parskip}    		% Activate to begin paragraphs with an empty line rather than an indent
\usepackage{graphicx}				% Use pdf, png, jpg, or eps§ with pdflatex; use eps in DVI mode
								% TeX will automatically convert eps --> pdf in pdflatex		
\usepackage{amssymb}
\usepackage{listings}

\title{Command Crib Sheet"}
\author{Abdul Alzaid}
%\date{}							% Activate to display a given date or no date

\begin{document}
\maketitle
%\section{}
%\subsection{}
This is a a Unix command glossary for at least ten commands. Each of which has, (1) A brief description of what each command does (2) A sample invocation of each command (3) A plain English description of what that invocation does.


\begin{center}
    \begin{tabular}{| l | l | l | l |}
    \hline
    Command & Description  \\ \hline
    cd & Change directory  \\ \hline
    ls &  ls provides a tree-like listing of a directory structure.  \\ \hline
    cp & Copy one or more files to another location  \\ \hline
    mv & Move or rename files or directories.  \\ \hline
    mkdir & Create new folder(s), if they do not already exist.  \\ \hline
    rmdir & Remove directory, this command will only work if the folders are empty.  \\ \hline
    rm & Remove files (delete/unlink)  \\ \hline
    man & Format and display help pages.  \\ \hline
    open & Open a file in its default application, using virtual terminal (VT)  \\ \hline
    cat &  an acronym for concatenate, lists a file to stdout.  \\ \hline
        \end{tabular}
\end{center}

{\Large Samples on how to use each command.}\\

("cd")
\begin{lstlisting} [frame=single]
~ AbdulZaid$ cd Desktop // Change directory to Desktop.
 \end{lstlisting} 
 
 ("ls")
 \begin{lstlisting} [frame=single]
Abduls-MacBook-Pro:~ AbdulZaid$ ls
2.28.circ	Desktop		GoideProjects	Music
Subtractor.circ    3.19.circ	Documents	Google Drive
Pictures   // it provides a list of wheat's in the directory.
 In this case Home dir
\end{lstlisting} 

("cp")
 \begin{lstlisting} [frame=single]
Abduls-MacBook-Pro:Documents AbdulZaid$ cp FinalEdit.pdf 
FirstEdit.pdf  // Copies a file to another file. in this case,
it copies the file "FinalEdit.pdf" to a new FirstEdit.pdf
\end{lstlisting} 

("mv")
 \begin{lstlisting} [frame=single]
Abduls-MacBook-Pro:rpg AbdulZaid$ mv index.html indexx.html
// mv moves or rename files, in this case it changes
index.html to indexx.html
\end{lstlisting} 


("mkdir")
 \begin{lstlisting} [frame=single]
Abduls-MacBook-Pro:rpg AbdulZaid$ mkdir newFolder // It makes
a new directory, where in this case the directory name is
 "newFolder"
\end{lstlisting} 

("rmdir")
 \begin{lstlisting} [frame=single]
Abduls-MacBook-Pro:rpg AbdulZaid$ rmdir newFolder // it remov
es the directory, where in this case the directory name is
 "newFolder"
\end{lstlisting} 

("rm")
 \begin{lstlisting} [frame=single]
Abduls-MacBook-Pro:rpg AbdulZaid$ rm indexx.html // rm 
removes the file typed after it. In this case it removes 
indexx.html
\end{lstlisting} 

("man")
 \begin{lstlisting} [frame=single]
Abduls-MacBook-Pro:rpg AbdulZaid$ man rm // it shows
a manual on how to use the command after it.  In this case
the command rm and its usage will be displayed on another
page.
\end{lstlisting} 

("open")
 \begin{lstlisting} [frame=single]
Abduls-MacBook-Pro:rpg AbdulZaid$ open index.html  // open,
open whatever file comes after it.  In this case it opens the 
file named index.html
\end{lstlisting} 

("cat")
 \begin{lstlisting} [frame=single]
Abduls-MacBook-Pro:rpg AbdulZaid$  cat index.html  // cat,
 concatenate and print files, and in this case it will print
 what's in the file to stdout.
\end{lstlisting} 

\end{document}  